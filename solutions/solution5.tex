\documentclass[fontsize=12pt]{article}

\usepackage{listings}
\usepackage[T1,T2A]{fontenc}
\usepackage[utf8]{inputenc}
\usepackage[russian]{babel}
\usepackage{amsmath}
\usepackage{xcolor}
\usepackage{scrextend}
\usepackage{hyperref}
\usepackage[left=2cm,right=2cm,top=2cm,bottom=2cm,bindingoffset=0cm]{geometry}
\usepackage{listings}
\usepackage{csquotes}
\usepackage{enumerate}
\usepackage{xcolor}
\usepackage{mathtools}

\hypersetup{
    colorlinks=true,
    linkcolor=cyan,
    filecolor=magenta,      
    urlcolor=blue,
}

\linespread{1.3}
\parindent=0.6cm
\lstset{tabsize = 2}

\DeclareMathOperator{\rank}{rank}
\makeatletter
\newenvironment{sqcases}{%
  \matrix@check\sqcases\env@sqcases
}{%
  \endarray\right.%
}
\def\env@sqcases{%
  \let\@ifnextchar\new@ifnextchar
  \left\lbrack
  \def\arraystretch{1.2}%
  \array{@{}l@{\quad}l@{}}%
}
\makeatother

\DeclarePairedDelimiter\floor{\lfloor}{\rfloor}

\title{Домашнее задание}
\author{Лев Довжик, M3439 \\ Вариант №62}
\date{}


\begin{document}
	\pagenumbering{gobble}
	\maketitle
	
\section*{Задача 3.1}

\subsection*{Коды Хэмминга}

Зафиксируем конкретное $r$. Тогда $n = 2^r - 1, k = 2^r - r - 1, d = 3$\\

Для двоичного кода граница Хэмминга имеет вид: $2^k \leq \dfrac{2^n}{\sum\limits_{i = 0}^t C_n^i}$\\ 

Немного преобразовав, получаем его в следующем виде: $\sum\limits_{i = 0}^t C_n^i \leq 2^{n - k}$\\

Так как $d = 3$, то $t = \floor*{\dfrac{3 - 1}{2}} = 1$, то есть код исправляет любые однократные ошибки. Тогда левая часть имеет вид: $\sum\limits_{i = 0}^1 C_n^i = C_n^0 + C_n^1 = 1 + n = 1 + (2^r - 1) = 2^r$\\

Справа же $2^{n - k} = 2^{2^r - 1 - (2^r - r - 1)} = 2^{2^r - 1 - 2^r + r + 1} = 2^r$

Как  видно обе части неравенства равны, а значит код Хэмминга удовлетврояет границе Хэмменга с равенством.

\subsection*{Коды Хэмминга}

$n = 23, k = 12, d = 7$\\

Для двоичного кода граница Хэмминга имеет вид: $2^k \leq \dfrac{2^n}{\sum\limits_{i = 0}^t C_n^i}$\\ 

Немного преобразовав, получаем его в следующем виде: $\sum\limits_{i = 0}^t C_n^i \leq 2^{n - k}$\\

Так как $d = 7$, то $t = \floor*{\dfrac{7 - 1}{2}} = 3$, то есть код исправляет любые трёхкратные ошибки. Тогда левая часть имеет вид: $\sum\limits_{i = 0}^3 C_n^i = C_{23}^0 + C_{23}^1 + C_{23}^2 + C_{23}^3 = 2048$\\

Справа же $2^{n - k} = 2^{23 - 12} = 2^{11} = 2048$

Как  видно обе части неравенства равны, а значит данный код Голея удовлетврояет границе Хэмменга с равенством.
\end{document}
 