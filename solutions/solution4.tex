\documentclass[fontsize=12pt]{article}

\usepackage{listings}
\usepackage[T1,T2A]{fontenc}
\usepackage[utf8]{inputenc}
\usepackage[russian]{babel}
\usepackage{amsmath}
\usepackage{xcolor}
\usepackage{scrextend}
\usepackage{hyperref}
\usepackage[left=2cm,right=2cm,top=2cm,bottom=2cm,bindingoffset=0cm]{geometry}
\usepackage{listings}
\usepackage{csquotes}
\usepackage{enumerate}
\usepackage{xcolor}
\usepackage{mathtools}

\hypersetup{
    colorlinks=true,
    linkcolor=cyan,
    filecolor=magenta,      
    urlcolor=blue,
}

\linespread{1.3}
\parindent=0.6cm
\lstset{tabsize = 2}

\DeclareMathOperator{\rank}{rank}
\makeatletter
\newenvironment{sqcases}{%
  \matrix@check\sqcases\env@sqcases
}{%
  \endarray\right.%
}
\def\env@sqcases{%
  \let\@ifnextchar\new@ifnextchar
  \left\lbrack
  \def\arraystretch{1.2}%
  \array{@{}l@{\quad}l@{}}%
}
\makeatother

\DeclarePairedDelimiter\floor{\lfloor}{\rfloor}

\title{Домашнее задание}
\author{Лев Довжик, M3439 \\ Вариант №62}
\date{}


\begin{document}
	\pagenumbering{gobble}
	\maketitle
	
\section*{Задача 2.1}

\subsection*{1) $R = \dfrac{1}{6}$}

В данном случае у на всего $2^1$ кодовых слова, так всего одно из них ненулевое, то вес второго будет определять $d$. Максимально возможный вектор $(1\ 1\ 1\ 1\ 1\ 1)$ с весом 6. 

Итого $G = (1\ 1\ 1\ 1\ 1\ 1)$

\subsection*{2) $R = \dfrac{2}{6}$}

Здесь у нас $2^2$ кодовых слова, три из которых ненулевые. Покажем порождающую матрицу, при которой $d = 4: 
G = \begin{pmatrix}
	 	1\ 1\ 1\ 0\ 1\ 0\\
		0\ 1\ 1\ 1\ 0\ 1		
	\end{pmatrix}$ 
	
При этом большего $d$ быть и не может, т.к. линейная комбинация(ведь кодовые слова образуют линейное пространство) двух шестимерных векторов веса 5, может быть либо равна нулевому вектору, либо вектору веса 2, а значит $d$ будет не более 2.

\subsection*{3) $R = \dfrac{3}{6}$}

Представим проверочную матрицу кода, который будет иметь $d = 3$. В ней все столбцы попарно различны, но $(1) + (2) = (3)$.

$H = \begin{pmatrix}
	 	0\ 1\ 1\ 0\ 0\ 1\\
		1\ 0\ 1\ 0\ 1\ 0\\
		1\ 1\ 0\ 1\ 0\ 0
	\end{pmatrix}$ 

Заметим, что это максимальное расстояние в силу того, что для $d = 4$ на нужно выбрать 6 двоичных векторов длины 3, так что бы как минимум любые три вектора были линейно независимы. Несложным перебором можно убедиться, что такое невозможно.

\subsection*{4) $R = \dfrac{4}{6}$}

Т.к. невозможно найти 6 различных двоичных векторов длины 2(столбцов проверочной матрицы), то максимальное $d = 2$ для такого кода. Представим же его проверочную матрицу:

$H = \begin{pmatrix}
	 	1\ 1\ 1\ 0\ 1\ 0\\
		1\ 1\ 0\ 1\ 0\ 1
	\end{pmatrix}$
	
\subsection*{5) $R = \dfrac{5}{6}$}
	
$H = (1\ 1\ 1\ 1\ 1\ 1)$ и $d$ так же равно 2. Больше же оно не может быть в силу границы Синглтона: $d \leq n - k + 1 = 2$

\subsection*{6) $R = \dfrac{6}{6}$}

Множество кодовых в данном случае --- это вообще все двоичные вектора длины 6. Минимальное расстояние в таком случае, очевидно, равно 1. Мы можем просто передавать исходное сообщение как есть ($G = I_6$).	

\section*{Задача 2.3}

\subsection*{Интерпритируем исходные матрицы как порожадющие}

а)$H = \begin{pmatrix}
1\ 1\ 0\ 1\ 0\ 0\\
1\ 1\ 0\ 0\ 1\ 0\\
1\ 0\ 1\ 0\ 0\ 1\\
\end{pmatrix} d = 3, R = \dfrac{1}{2}$
{ } { } { }  { } { } { }
б)$H = \begin{pmatrix}
1\ 1\ 0\ 1\ 0\ 0\\
1\ 1\ 0\ 0\ 1\ 0\\
1\ 0\ 1\ 0\ 0\ 1\\
\end{pmatrix} d = 3, R = \dfrac{1}{2}$\\\\

\noindent
в)$H = \begin{pmatrix}
1\ 0\ 1\ 0\ 0\ 0\\
1\ 1\ 0\ 1\ 0\ 0\\
0\ 1\ 0\ 0\ 1\ 0\\
1\ 0\ 0\ 0\ 0\ 1\\
\end{pmatrix} d = 3, R = \dfrac{1}{3}$
{ } { } { }  { } { } { }
г)$H = \begin{pmatrix}
0\ 1\ 1\ 1\ 0\ 0\\
1\ 1\ 1\ 0\ 1\ 0\\
0\ 0\ 1\ 0\ 0\ 1\\
\end{pmatrix} d = 3, R = \dfrac{1}{2}$\\\\

\noindent
д)$H = \begin{pmatrix}
0\ 1\ 0\ 1\ 0\ 0\\
1\ 1\ 1\ 0\ 1\ 0\\
0\ 0\ 1\ 0\ 0\ 1\\
\end{pmatrix} d = 2, R = \dfrac{1}{2}$
{ } { } { }  { } { } { }
е)$H = \begin{pmatrix}
1\ 0\ 0\ 1\ 0\ 0\\
1\ 1\ 1\ 0\ 1\ 0\\
1\ 1\ 1\ 0\ 0\ 1\\
\end{pmatrix} d = 2, R = \dfrac{1}{2}$\\\\

\noindent
ж)$H = \begin{pmatrix}
1\ 0\ 1\ 0\ 0\ 0\\
0\ 1\ 0\ 1\ 0\ 0\\
0\ 1\ 0\ 0\ 1\ 0\\
1\ 0\ 0\ 0\ 0\ 1\\
\end{pmatrix} d = 3, R = \dfrac{1}{3}$
{ } { } { }  { } { } { }
з)$H = \begin{pmatrix}
1\ 0\ 1\ 0\ 0\ 0\ 0\\
0\ 1\ 0\ 0\ 1\ 0\ 0\\
1\ 0\ 0\ 0\ 0\ 1\ 0\\
1\ 1\ 0\ 1\ 0\ 0\ 1\\
\end{pmatrix} d = 3, R = \dfrac{3}{7}$\\\\

\noindent
и)$H = \begin{pmatrix}
1\ 0\ 1\ 1\ 0\ 0\ 0\\
1\ 1\ 1\ 0\ 1\ 0\ 0\\
1\ 1\ 0\ 0\ 0\ 1\ 0\\
0\ 1\ 1\ 0\ 0\ 0\ 1\\
\end{pmatrix} d = 4, R = \dfrac{1}{3}$
{ } { } { }  { } { } { }
к)$H = \begin{pmatrix}
0\ 1\ 1\ 0\ 0\ 0\ 0\\
1\ 0\ 0\ 0\ 1\ 0\ 0\\
0\ 0\ 0\ 1\ 0\ 1\ 0\\
1\ 1\ 0\ 1\ 0\ 0\ 1\\
\end{pmatrix} d = 3, R = \dfrac{3}{7}$\\\\

\subsection*{Итерпритируем исходные матрицы как проверочные}

а)$G = \begin{pmatrix}
1\ 0\ 0\ 1\ 1\ 1\\
0\ 1\ 0\ 0\ 1\ 1\\
1\ 0\ 1\ 1\ 1\ 0\\
\end{pmatrix} d = 3, R = \dfrac{1}{2}$
{ } { } { }  { } { } { }
б)$G = \begin{pmatrix}
1\ 0\ 0\ 1\ 1\ 1\\
0\ 1\ 0\ 0\ 1\ 1\\
0\ 0\ 1\ 1\ 1\ 0\\
\end{pmatrix} d = 3, R = \dfrac{1}{2}$\\\\

\noindent
в)$G = \begin{pmatrix}
1\ 0\ 0\ 0\ 0\ 1\\
0\ 1\ 0\ 0\ 1\ 0\\
0\ 0\ 1\ 0\ 0\ 1\\
0\ 0\ 0\ 1\ 1\ 1\\
\end{pmatrix} d = 3, R = \dfrac{2}{3}$
{ } { } { }  { } { } { }
г)$G = \begin{pmatrix}
1\ 0\ 0\ 1\ 1\ 0\\
0\ 1\ 0\ 1\ 0\ 1\\
0\ 0\ 1\ 0\ 0\ 1\\
\end{pmatrix} d = 2, R = \dfrac{1}{2}$\\\\

\noindent
д)$G = \begin{pmatrix}
1\ 0\ 0\ 1\ 1\ 1\\
0\ 1\ 0\ 1\ 0\ 0\\
0\ 0\ 1\ 0\ 0\ 1\\
\end{pmatrix} d = 2, R = \dfrac{1}{2}$
{ } { } { }  { } { } { }
е)$G = \begin{pmatrix}
1\ 0\ 0\ 1\ 0\ 0\\
0\ 1\ 1\ 1\ 0\ 1\\
0\ 0\ 0\ 0\ 1\ 1\\
\end{pmatrix} d = 2, R = \dfrac{1}{2}$\\\\

\noindent
ж)$G = \begin{pmatrix}
1\ 0\ 0\ 0\ 0\ 1\\
0\ 1\ 0\ 0\ 1\ 0\\
0\ 0\ 1\ 0\ 0\ 1\\
0\ 0\ 0\ 1\ 1\ 0\\
\end{pmatrix} d = 2, R = \dfrac{2}{3}$
{ } { } { }  { } { } { }
з)$G = \begin{pmatrix}
1\ 0\ 0\ 0\ 0\ 1\ 0\\
0\ 1\ 0\ 0\ 1\ 0\ 0\\
0\ 0\ 1\ 0\ 0\ 1\ 0\\
0\ 0\ 0\ 1\ 1\ 1\ 1\\
\end{pmatrix} d = 2, R = \dfrac{4}{7}$\\\\

\noindent
и)$G = \begin{pmatrix}
1\ 0\ 0\ 0\ 1\ 0\ 1\\
0\ 1\ 0\ 0\ 1\ 1\ 1\\
0\ 0\ 1\ 0\ 1\ 1\ 0\\
0\ 0\ 0\ 1\ 0\ 1\ 1\\
\end{pmatrix} d = 3, R = \dfrac{4}{7}$
{ } { } { }  { } { } { }
к)$G = \begin{pmatrix}
1\ 0\ 0\ 0\ 1\ 0\ 0\\
0\ 1\ 0\ 0\ 1\ 1\ 1\\
0\ 0\ 1\ 0\ 1\ 1\ 1\\
0\ 0\ 0\ 1\ 0\ 1\ 0\\
\end{pmatrix} d = 2, R = \dfrac{4}{7}$\\\\


\section*{Задача 2.11}

Предложим код с $d = 5, k = 3, n = 10 < 5 \cdot 3 = 15$. Для этого будем порождать его из $g = (1\ 1\ 1\ 0\ 1\ 0\ 0\ 1\ 0\ 0)$.\\

$G = \begin{pmatrix}
1\ 1\ 1\ 0\ 1\ 0\ 0\ 1\ 0\ 0\\
0\ 1\ 1\ 1\ 0\ 1\ 0\ 0\ 1\ 0\\
0\ 0\ 1\ 1\ 1\ 0\ 1\ 0\ 0\ 1\\
\end{pmatrix}$\\

Докажем, что его минимальное расстояние равно 5. Для этого переберём все ненулевые сообщения и найдём минимальный вес соответствующих кодовых слов:

\begin{itemize}
\item
$m = (0\ 0\ 1) \Rightarrow mG = (0\ 0\ 1\ 1\ 1\ 0\ 1\ 0\ 0\ 1) \Rightarrow w = 5$ 

\item
$m = (0\ 1\ 0) \Rightarrow mG = (0\ 1\ 1\ 1\ 0\ 1\ 0\ 0\ 1\ 0) \Rightarrow w = 5$

\item
$m = (0\ 1\ 1) \Rightarrow mG = (0\ 1\ 0\ 0\ 1\ 1\ 1\ 0\ 1\ 1) \Rightarrow w = 6$

\item
$m = (1\ 0\ 0) \Rightarrow mG = (1\ 1\ 1\ 0\ 1\ 0\ 0\ 1\ 0\ 0) \Rightarrow w = 5$

\item
$m = (1\ 0\ 1) \Rightarrow mG = (1\ 1\ 0\ 1\ 0\ 0\ 1\ 1\ 0\ 1) \Rightarrow w = 6$

\item
$m = (1\ 1\ 0) \Rightarrow mG = (1\ 0\ 0\ 1\ 1\ 1\ 0\ 1\ 1\ 0) \Rightarrow w = 6$

\item
$m = (1\ 1\ 1) \Rightarrow mG = (1\ 0\ 1\ 0\ 0\ 1\ 1\ 1\ 1\ 1) \Rightarrow w = 7$
\end{itemize} 

Как видно минимальный вес равен 5, а значит ему и равно минимальное расстояние между кодовыми словами.

\section*{Задача 2.12}

\subsection*{a) g = (1 1 0 1)}

$G = \begin{pmatrix}
1\ 1\ 0\ 1\ 0\ 0\ 0\ 0\ 0\ 0\\
0\ 0\ 1\ 1\ 0\ 1\ 0\ 0\ 0\ 0\\
0\ 0\ 0\ 0\ 1\ 1\ 0\ 1\ 0\ 0\\
0\ 0\ 0\ 0\ 0\ 0\ 1\ 1\ 0\ 1\\
\end{pmatrix} n = 10, R = \dfrac{2}{5}, d = 3$\\
 
Лучший же $(10, 4)$-код, по данным \href{http://www.codetables.de}{codetables.de}, имеет минимальное расстояние $4$. 

\subsection*{б) g = (1 1 0 1 0 1)}

$G = \begin{pmatrix}
1\ 1\ 0\ 1\ 0\ 1\ 0\ 0\ 0\ 0\\
0\ 0\ 1\ 1\ 0\ 1\ 0\ 1\ 0\ 0\\
0\ 0\ 0\ 0\ 1\ 1\ 0\ 1\ 0\ 1\\
\end{pmatrix} n = 10, R = \dfrac{3}{10}, d = 4$\\
 
Лучший же $(10, 3)$-код, по данным \href{http://www.codetables.de}{codetables.de}, имеет минимальное расстояние $5$. 

\subsection*{в) g = (1 1 0 1 1 1)}

$G = \begin{pmatrix}
1\ 1\ 0\ 1\ 1\ 1\ 0\ 0\ 0\ 0\\
0\ 0\ 1\ 1\ 0\ 1\ 1\ 1\ 0\ 0\\
0\ 0\ 0\ 0\ 1\ 1\ 0\ 1\ 1\ 1\\
\end{pmatrix} n = 10, R = \dfrac{3}{10}, d = 5$\\
 
 Минимального расстояние этого кода совпадает с таковым для лучшего $(10, 3)$-кода, по данным \href{http://www.codetables.de}{codetables.de}.

\end{document}
 